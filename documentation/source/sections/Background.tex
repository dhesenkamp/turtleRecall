\section{Background}

Three-fifths of all turtle species worldwide are on the verge of extinction or are already severely threatened. According to a U.S. study published in the journal BioScience, this makes turtles the most endangered vertebrates in the world, ahead of mammals, birds, fish, and amphibians.

Sea turtles are known as indicator species which means that their presence and abundance reflect the health of the wider ecosystem. Therefore, increasing our ability to identify and understand them can enhance our ecological understanding.

Not only for their own sake, but they also contribute significantly to a healthy ecosystem. The herbivores, carnivores, or omnivores are at the same time hunters, pest controllers, and food sources for other animals. The scavenging species, for example, ensure a clean environment, and the herbivorous turtles make an important contribution to spreading plant seeds.

For many sea turtles, sea grass is the main food source. Sea grass grows in thick beds on shallow seabeds. Constant feeding by sea turtles on this grass keeps the beds in order and prevents them from becoming long and unhealthy. Because these sea grass beds are prime locations for small fish to breed and spawn, healthy sea grass beds are critical to populations of small fish living in the oceans. Without this contribution from sea turtles, the ocean ecosystem would be out of balance.

While sea turtles spend most of their lives in the ocean, they come to the beach to lay their eggs. This important part of a turtle's life also has an important impact on a beach's ecosystem. Without plants like beach grasses, the beach would succumb to erosion. These plants are fertilized by eggs that do not hatch and by turtle droppings on the beach. This nutrition is essential for the survival of the beach ecosystem.

Humans are making it difficult for the turtles to survive. The animals are suffering from climate change, habitat destruction, excessive trade in live animals, the sale of their meat and shells, and environmental pollution.
For example, Australian scientists estimated that one in five sea turtles now dies from eating too much plastic. With 14 swallowed plastic particles, the risk of death increases to 50 percent. At 200 pieces swallowed, the sea turtle is no longer viable. The plastic pieces get stuck or cause internal injuries.

The greatest threat to turtles, however, is and remains humans. In some regions of the world, turtles are considered a delicacy; worldwide, they are kept as exotic pets. The meat of larger species ends up on markets in Asia, Africa, and Latin America, for example, while smaller species are traded internationally primarily as pets. The turtle shell ends up in the powdered form in pills and pastes used in traditional Asian medicine (including TCM). Some species, which are said to have a special medicinal effect, such as the three-striped hinge back turtle (Cuora trifasciata) achieve prices of several thousand US dollars per animal. Even in Europe, turtles can still be ingredients in TCM formulations.