\section{Introduction}

Three-fifths of all turtle species worldwide are on the verge of extinction or are already severely threatened \citep{Paladino2001, Goldstein2021, Mortimer2008}. This makes turtles the most endangered vertebrates in the world, ahead of mammals, birds, fish, and amphibians.

Sea turtles are known as indicator species which means that their presence and abundance reflect the health of the wider ecosystem. Therefore, increasing our ability to identify and understand them can enhance our ecological understanding. The herbivores, carnivores, or omnivores are at the same time hunters, pest controllers, and food sources for other animals. The scavenging species, for example, ensure a clean environment, and the herbivorous turtles make an important contribution to spreading plant seeds. They maintain beds of seagrass and coral reefs and facilitate a nutrient cycle from marine to land ecosystems. Humans are making it difficult for the turtles to survive. The animals are suffering from climate change, habitat destruction, excessive trade in live animals, the sale of their meat and shells, and environmental pollution.

Deepmind \footnote{\url{https://deepmind.com/}}, Ocean Hub Africa \footnote{\url{https://ocean-innovation.africa/}}, and Local Ocean Conservation \footnote{\url{http://localocean.co/}} have come together to provide a dataset of sea turtle images and a challenge to built a machine learning model for identification of sea turtles by their facial scales. We use this dataset to built a convolutional neural network (CNN) that is supposed to do exactly that: recognise known turtles via their unique head patterns from images or classify them as new, so far unknown individuals.