\section{Grant Species Preservation Through Population Census}


The ability to distinguish between individuals of the same species is a fundamental tool for modern animal welfare. To ensure the protection of individuals, it is crucial to identify their whereabouts and movement patterns. Because sea turtles are a powerful indicator of overall ecosystem health, accurate identification serves to enhance our ecological understanding. Implicitly, this means that ensuring species conservation can be generated and optimized.  


\subsection{Turtle Tagging}
In the past, the detection and tracking of individuals was done by attaching tags to the fins of the individuals found. This method is severely compromised by the loss of tags and thus the successful tracking of population dynamics is not guaranteed. Basically, the use of \emph{flipper tags} was common, which are mostly made of metal and plastic. This method is very costly due to the extraordinarily long life span of the turtles, which means that the flipper tags were lost or identification was no longer possible due to wear and tear of the material.


\subsection{Photographic Identification}
Just as the human finger print, turtles have unique and time-stable facial scales by which they can be identified \cite{Carpentier2016}. Photographic identification has become the method of choice over time as it is non-invasive and low-cost. Due to the advances in machine learning and object identification, we are able to identify turtle individuals with algorithms as opposed to humans manually searching through a database of images. The goal of such machine learning algorithms is to assign a unique ID to each individual if it already exists in the database and to create a new ID if a new individual has been sighted.